%%%%%%%%%%%%%%%%%%%%%%%%%%%%%%%%%%%%%%%%%%%%%%%%%%%%%%%%%%%%%%%%%%%%%%%%%%%%%%%%%%%%
% Do not alter this block (unless you're familiar with LaTeX
\documentclass{article}
\usepackage[utf8]{inputenc} % UTF8 юникод текст оруулах
\usepackage[T2A]{fontenc} % кирил үсгийн кодчилол
\usepackage[mongolian]{babel} % олон хэлний текст
\usepackage[margin=1in]{geometry}
\usepackage{amsmath,amsthm,amssymb,amsfonts, fancyhdr, color, comment, graphicx, environ}
\usepackage{xcolor}
\usepackage{mdframed}
\usepackage[shortlabels]{enumitem}
\usepackage{indentfirst}
\usepackage{hyperref}
\hypersetup{
    colorlinks=true,
    linkcolor=blue,
    filecolor=magenta,
    urlcolor=blue,
}

\pagestyle{fancy}

\newenvironment{problem}[2][Бодлого]
    { \begin{mdframed}[backgroundcolor=gray!20] \textbf{#1 #2} \\}
    {  \end{mdframed}}

% Define solution environment
\newenvironment{solution}{\textbf{Бодолт:}}

\newcounter{example}[section]
\newenvironment{example}[1][]{\refstepcounter{example}\par\medskip
   \noindent \textbf{Жишээ~\theexample. #1} \rmfamily}{\medskip}

\newtheorem{definition}{Тодорхойлолт}[section]

%%%%%%%%%%%%%%%%%%%%%%%%%%%%%%%%%%%%%%%%%%%%%
%Fill in the appropriate information below
\lhead{F.CS309 Тооцооллын хүндрэл \\ СЕМИНАР 1} % Оюутны код, Оюутны нэр
\rhead{2021-2022 Намар \\ Багш Г.Ганбат}
%%%%%%%%%%%%%%%%%%%%%%%%%%%%%%%%%%%%%%%%%%%%%


\begin{document}
\section*{Удирдамж}
\begin{itemize}
  \item Семинарын цаг дэр сэдвийн ойлголтыг хэлэлцэж, дасгалууд дээр ажиллана.
  \item Гэрийн даалгавруудыг хичээлийн бус цагаараа шийдэж, \textbf{2022 оны 9 сарын 9-ны 23:59} цагаас өмнө Moodle ситсемд илгээсэн байна.
  \item Гэрийн даалгавруудыг бэлтгэх
  \begin{enumerate}
    \item Шийдлээ гүйцэд, тодорхой, хангалттай дэлгэрэнгүй тайлбарлана. Оюутны нэр, код, Бодлогын даалгавар, Бодолт гэсэн бүтэцтэй байна.
    \item Latex дээр боловсруулах бөгөөд нэг ширхэг PDF файл болгон илгээнэ. Загвар файлыг Moodle системээс татан авч ашиглана.
    \item Бусадтай зөвлөлдөж болно. Гэхдээ өөрийн зүгээс хандаж, өөрийн үгээр тайлбарлаж бичээрэй
  \end{enumerate}
\end{itemize}
\section*{Сэдвийн ойлголт, дасгалууд}
\subsection{Decision problem / Хэл }
Тэмдэг мөрүүдийг тэмдэгт мөрүүдэд харгалзуулах функцүүийн чухал тусгай тохиолдол бол гаралт нь ганцхан бит байдаг Буулийн функцүүд (Boolean functions) юм. Ийм $f$ функцийг $\{0, 1\}^*$-ийн $L_f = \{x: f (x) = 1\}$ дэд олонлогоор тодорхойлж тэдгээрийг \textit{хэлнүүд} эсвэл  \textit{шийдвэрийн асуудал (decision problem)-уудын} олонлог гэдэг. $f$ функцийг бодох (өгөгдсөн $x$-ээр $f(x)$-ыг бодох) тооцооллын асуудлыг $L_f$ хэлийг шийдэх асуудалтай (өгөгдсөн $x$-ийн хувьд $x\in L_f$ байгаа эсэхийг шийдэх) хамтад нь тодорхойлдог.

\begin{example}
Үдшийн зоогт уригдсан хүмүүсийг графын оройнууд гэж үзээд, биенээ танихгүй (don't get along) дурын хоёр хүнийг ирмэгээр холбоё. Тус графаас хамгийн их хэмжээтэй \textit{бие даасан олонлог} (ямар ч нийтлэг ирмэггүй оройн олонлог)-ийг олох асуудал нь үдшийн зоогийн тооцооллын асуудал юм.

Танилцуулгын оройн зоогийн үдэшлэгт тавигдах асуудал нь өгөгдсөн графикаас хамгийн их хэмжээтэй бие даасан багц (нийтлэг ирмэггүй оройн багц) олох асуудал болдог. Үүнийг илэрхийлэх хэл нь дараах байдлаар томъёологдоно:
$$INDSET = \{\langle G, k \rangle : \exists S \subseteq V(G) \text{ энд } \mid S \mid \geq k \text{ ба } \forall u, v \in  S, \overline{uv} \notin E(G)\}$$
Энэ хэлийг шийдвэрлэх алгоритм нь $G$ граф ба $k$ тоог оруулснаар \textit{бие даасан олонлог} гэж нэрлэгддэг, хамгийн багадаа $k$ хэмжээтэй хоорондоо хамааралгүй оролцогчдын олонлог байгаа эсэхийг тодорхойлно.
\end{example}

\subsection{Формаль хэл}

$G =(V, T, S, P)$ phrase-structure grammar байг. Тэгвэл $G$-ээр үүсгэдэх $L(G)$ гэсэн хэлний бүх тэмдэгт мөрүүд, үгүүд нь эхлэлийн S төлвөөс гаргалгаатай.
$L(G) = \{w \in T* \mid S  \dot{\Rightarrow} w\}$.

\begin{example}
$V = \{S, A, a, b\}$ үгсийн сан, $T = \{a, b\}$ терминал үгийн олонлог, $S$ эхлэлийн тэмдэгт болон бүтээгдэхүүн үүсгэлт нь $P = \{S \rightarrow aA, S \rightarrow b, A \rightarrow aa \}$ гэж өгөгдсөн $G$ граф байг.
$L(G) = \{b, aaa\}$, because we can begin a derivation with $S \rightarrow aA$ or with $S \rightarrow b$, and  from $aA$ we can derive $aaa$ using  $A \rightarrow aa$. There are no other possible derivations.
\end{example}

\subsection{Big-Oh тэмдэглэгээ}
Алгоритмын тооцоолох үр ашгийг энгийн үйлдлүүдийн тоогоор хэмждэг бөгөөд \textit{оролтын уртын функц}-ээр илэрхийлдэг. Алгоритмын үр ашгийг натурал тоон $\mathbb{N}$ олонлогоор тодорхойлогдох $T$ функцээр илэрхийлж болох бөгөөд ингэснээр $T(n)$ нь алгоритмын $n$ урттай оролт дээр гүйцэтгэдэг энгийн үйлдлүүдийн хамгийн их тоог илэрхийлнэ. Гэвч $T$ функц нь энгийн үйлдлүүдийг тоолоход доод түвшний нарийвчлалаас хэт хамааралтай байдаг. Жишээлбэл, нэмэх алгоритм нь хэрэв нэг оронтой \textit{хоёрт}-ын тоонуудыг нэмдэг бол \textit{аравт}-ын тоотой харьцуулахад гурав дахин их үйлдэл хийх болно. Эдгээр доод түвшний нарийвчлалыг орхиж, илүү том зураг дээр анхаарлаа хандуулахад дараах алдартай тэмдэглэгээг хэрэглэдэг.

\begin{definition}[Big-Oh тэмдэглэгээ]
Хэрэв $f, g$ нь $\mathbb{N}$-ээс $\mathbb{N}$ хүртэлх хоёр функц бол (1) хангалттай том $n$ бүрт $f (n) \leq c \cdot g(n)$ байх $c$ тогтмол олддог бол $f = O(g)$, (2) хэрэв $g = O(f)$ байвал $f = \Omega(g)$, (3) $f = O(g)$ ба $g = O(f)$ нь $f = \Theta(g)$, (4) хангалттай том $n$ бүрийн хувьд $\epsilon> 0, f(n)\leq\epsilon\cdot g(n)$ байвал $f=o(g)$, (5)$g=o(f)$ бол $f=\omega(g)$ гэж хэлнэ.
\end{definition}

\begin{example}
    Big-Oh тэмдэглэгээ хэрэглэсэн жишээнүүд:
  \begin{enumerate}
    \item Хэрэв $f(n) = 100n \log n$ ба $g(n) = n^2$ байвал $f = O(g), g = \Omega(f), f = o(g), g = \omega(f)$ гэсэн харьцааг биелнэ гэж үзнэ.
    \item Хэрэв $f(n) = 100n^2 + 24n + 2 \log n$ ба $g(n) = n^2$ бол $f = O(g)$ байна. Энэ харьцааг $f(n) = O(n^2)$ гэж бичнэ. Мөн $g = O(f)$ тул $f=\Theta(g)$ ба $g=\Theta(f)$ гэж үзнэ.
    \item Хэрэв $f (n) = \min \{n, 10^6\}$ ба $g (n) = 1$ бол $n$ бүрт $f = O (g)$ байна. Үүнийг илэрхийлэхийн тулд бид $f = O (1)$ тэмдэглэгээг ашигладаг. Үүний нэгэн адил, хэрэв $n$-ээс хамаарсан хязгааргүй рүү дөхдөг $h$ функцийн (өөрөөр хэлбэл $n$ нь хангалттай их утгатай байх үед $c$ бүр нь $h(n)>c$ байдаг) хувьд $h = \omega(1)$ гэж бичнэ.
    \item Хэрэв $f (n) = 2^n$ бол $c\in\mathbb{N}$ тоо бүрийн хувьд, хэрэв $g(n) = n^c$ бол $g = o(f)$ болно. Бид үүнийг заримдаа $2^n = n^{\omega(1)}$ гэж бичдэг. Үүний нэгэн адил $h (n) = n^{O(1)}$ гэж бичээд $h$ нь дээрээсээ ямар нэг олон гишүүнтээр хязгаарлагддаг болохыг илэрхийлдэг. Өөрөөр хэлбэл хангалттай том $n, h(n)\leq{n^c}$ байх $c> 0$ тоо байна гэж үзнэ. Бид энэ тохиолдолд $h(n) = poly(n)$ гэж бас бичиж болно.
  \end{enumerate}
\end{example}

\section*{Гэрийн даалгаврууд}
\begin{enumerate}
  \item Дараах $f, g$ хос функц бүрийн хувьд $f = o(g), g=o(f), f =\Theta(g)$ байх эсэхийг тодорхойл. Хэрэв $f=o(g)$ бол $f(n)<g(n)$ нөхцөлийг хангах хамгийн эхний $n$ тоог ол:
        \begin{enumerate}[a)]
            \item $f(n) = n^2, g(n) = 2n^2 + 100\sqrt{n}.$
            \item $f(n) = n^{100}, g(n) = 2^{n/100}.$
            \item $f(n) = 1000n, g(n) = n \log n.$
        \end{enumerate}
  \item Дараах рекурсив функц бүрийн хувьд $f (n) = \Theta(g (n))$ байх $g$ функцийн төгсгөлийн (рекурсив биш) илэрхийллийг олж, тэр тохиолдолд батал. (Жич: Доор зөвхөн рекурсив дүрмийг өгсөн. $f (1) = f (2) = \cdots = f (10) = 1$ шалгах ба $n>10$-ийн үед рекурсив дүрмийг хэрэглэсэн ч болно;)
        \begin{enumerate}[a)]
            \item $f(n) = f(n-1) + 10.$
            \item $f(n) = f(n-1) + n.$
            \item $f(n) = 2f(n-1).$
        \end{enumerate}
  \item MIT музейд Arthur Ganson-ы \textit{Machine with Concrete} нэртэй кинетик үзмэр байдаг (Зураг 3-ыг үзнэ үү). Энэ нь бие биентэйгээ дараалан холбогдсон 13 араанаас бүрдэх бөгөөд араа бүр өмнөхөөсөө 50 дахин удаан хөдөлдөг. Хамгийн хурдан араа нь хөдөлгүүрээр нэг минутад 212 удаа эргэлддэг. Хамгийн удаан араа нь бетон дээр бэхлэгдсэн тул огт хөдөлж чадахгүй. Энэ машин яагаад эвдрэхгүй байгааг тайлбарлана уу.
\end{enumerate}

\begin{figure}[ht!]
    \centering
    \includegraphics[width=\textwidth]{agmachine.jpg}
    \caption{Machine with Concrete кинетик үзмэр.}
\end{figure}

\end{document} 